\documentclass{article}
%\usepackage[provide=*]{babel}
\usepackage{paracol}
%\twosided[pc]
% \footnotelayout{p}
%\usepackage{lipsum}
\usepackage[spanish]{babel}
\usepackage[twoside]{geometry}
\usepackage{sectsty}
\geometry{left=20mm, right=20mm}
\tolerance=1
\emergencystretch=\maxdimen
\hyphenpenalty=10000
\hbadness=10000
\sectionfont{\fontsize{12}{15}\selectfont}
\subsectionfont{\fontsize{10}{15}\selectfont}

\begin{document}
\begin{paracol}{3} % 
  \begin{nthcolumn*}{0} % Título [la]
    \subsection*{\centering DISPVTATIO LXIII}
    \subsection*{\centering De pretio rerum vendibilium} 
  \end{nthcolumn*}
  \vspace{0.5cm}
  \begin{nthcolumn}{1} % Título [es] Deepseek
    \subsection*{\centering DISPUTACIÓN 63}
    \subsection*{\centering Acerca del precio de las cosas vendibles.}
    \subsection*{\centering \emph{Deepseek}}
  \end{nthcolumn}
  \vspace{0.5cm}
  \begin{nthcolumn}{2} % Título [es] Mi traducción
    \subsection*{\centering DISPUTACIÓN 63}
    \subsection*{\centering Acerca del precio de las cosas vendibles.}
    \subsection*{\centering \emph{Mi traducción}}
  \end{nthcolumn}
  \vspace{0.5cm}
  \begin{nthcolumn*}{0} % [la]
    Materiam huius contractus esse diximus ex parte vendentis rem vendibilem, seu mercem, cuius dominium in emptorem transfert; ex parte vero emptoris pretium mercis, cuius dominium pariter in venditorem transfundit: diximus de merce, seu re vendibili, dicamus modo de pretio; quia tota huius contractus iustitia consistit in pretio cum merce coaequando. Ita vt valor mercis, et pretii vnus, idemque fit. Defudabit ergo huius disputationis labor hoc iusto pretio explicando, et cum re venali coaequando.
  \end{nthcolumn*}
  \vspace{0.5cm}
  \begin{nthcolumn}{1} % [es] Deepseek
    Hemos dicho que la materia de este contrato consiste, por parte del vendedor, en una cosa vendible, o mercancía, cuyo dominio transfiere al comprador; y por parte del comprador, en el precio de la mercancía, cuyo dominio igualmente transfiere al vendedor. Hemos hablado de la mercancía, o cosa vendible; hablemos ahora del precio, porque toda la justicia de este contrato consiste en equiparar el precio con la mercancía. De modo que el valor de la mercancía y el precio sean uno y el mismo. Por tanto, el esfuerzo de esta discusión se afanará en explicar este precio justo y en equipararlo con el bien vendible.
  \end{nthcolumn}
  \vspace{0.5cm}
  \begin{nthcolumn}{2} % [es] Mi traducción
    La materia de este contrato, dijimos ser, por parte del que vende, una cosa vendible o mercancía, de la cual, el dominio transfiere al comprador; mientras por parte del comprador, el precio de la mercancía, del cual, el dominio transfiere igualmente al vendedor. Hablamos acerca de la mercancía o de la cosa vendible, hablemos ahora acerca del precio; pues toda la justicia de este contrato consiste en igualar el precio con la mercancía. De modo tal que el valor de la mercancia y del precio sea uno y el mismo. Por lo tanto, el trabajo de esta discusión se esforzará en explicar este precio justo y en igualarlo con la cosa vendible.
  \end{nthcolumn}
  \vspace{0.5cm}
  \begin{nthcolumn*}{0} % [la] Título Sección I
    \subsection*{\centering Sectio I}
    \subsection*{\centering Quot sint pretia rerum vendibilium.} 
  \end{nthcolumn*}
  \vspace{0.5cm}
  \begin{nthcolumn}{1} % [es] Título Sección I [Deepseek]
    \subsection*{\centering Sección I}
    \subsection*{\centering ¿Cuántos son los precios de los artículos vendibles?}
  \end{nthcolumn}
  \vspace{0.5cm}
  \begin{nthcolumn}{2} % [es] Título Sección I [Mi traducción]
    \subsection*{\centering Sección I}
    \subsection*{\centering }
  \end{nthcolumn}
  \vspace{0.5cm}
  \begin{nthcolumn*}{0} % [la] Par_1 Sección I
    Priusquam ad divisionem pretii veniamus, definiendum est nobis pretium in genere, licet in nullo Doctore eius repererim definitionem. Pretium latissime dicitur, et strictissime: latissime, dictum transcendit nostrum propositum, et materiam emptionum et venditionum, de qua agimus, et invenitur in omnibus rebus tam materialibus quam spiritualibus, et nihil est aliud quam aestimatio cuiusque rei. Quae aestimatio est rectum iudicium rerum, quo alias aliis conferentes, et quasi iudicii nostri fideli trutina ponderantes, alias aliis vel postponimus, vel praeponimus, vel aequamus secundum cuiuscumque rei meritum et dignitatem. Sicut enim is, qui aqua lance propendit rerum pondera et momenta, ad dignoscendum pretium et pondus eius, alijs ponderibus rem illam praeponit, alijs postponit, alijs coaequat, iis scilicet, quae tantundem valoris habent. Ita nos iudicio nostro res aestimantes ponderantes, ac trutinantes alias aliis postponimus, alias anteferimus, alias coaequamus; et illud iudicium, quo eas aestimamus, est pretium, in quo easdem res habemus, et hoc modo tot sunt pretia rerum, quot sunt hominum, immo et Angelorum, et ipsius Dei iudicia de illis.
  \end{nthcolumn*}
  \vspace{0.5cm}
  \begin{nthcolumn}{1} % [es] Par_1 Sección I [Deepseek]
    Antes de abordar la división del precio, debemos definir el precio en general, aunque no he encontrado su definición en ningún Doctor. El precio se entiende en un sentido muy amplio y en otro muy restringido: en sentido amplio, dicho término trasciende nuestro propósito y la materia de compras y ventas sobre la cual tratamos, y se encuentra en todas las cosas, tanto materiales como espirituales, y no es otra cosa que la estimación de cada cosa. La estimación es el juicio recto de las cosas, mediante el cual, al comparar unas con otras y, por así decirlo, sopesarlas con la fiel balanza de nuestro criterio, unas a otras las posponemos, preferimos o igualamos según el mérito y dignidad de cada cosa. Porque así como aquel que, con una balanza de agua, equilibra los pesos y medidas de las cosas para determinar su precio y peso, a unos pesos antepone esa cosa, a otros la pospone, a otros la iguala, es decir, a aquellos que tienen el mismo valor. Así nosotros, con nuestro juicio, valorando, ponderando y sopesando las cosas, unas a otras posponemos, unas anteponemos, unas igualamos; y aquel juicio, por el cual las valoramos, es el precio, en el cual tenemos a esas mismas cosas, y de este modo tantos son los precios de las cosas, cuantos son los juicios de los hombres, más aún de los Ángeles, y los juicios del mismo Dios sobre ellas.
  \end{nthcolumn}
  \vspace{0.5cm}
  \begin{nthcolumn}{2} % [es]  Par_1 Sección I [Mi traducción]
    Antes que abordemos la división del precio, debe ser definido por nosotros el precio en general, aunque en ningún Doctor haya encontrado una definición de ello. El [término] precio se emplea en un sentido amplio y en uno estricto: en el amplio, el término trasciende nuestro propósito y la materia de compra y venta, sobre la cual tratamos, y se encuentra en todas las cosas tanto materiales como espirituales, y no es distinto que la \emph{estimación} de cada cosa. La cual es el recto juicio de las cosas, en que mientras unas con otras se comparan y, por así decirlo, se ponderan con la fiel balanza de nuestro juicio, o bien, postergamos, anteponemos o igualamos según el mérito y la dignidad de cada cosa.
  \end{nthcolumn}
\end{paracol}
\end{document}